\documentclass[sigconf]{acmart}

\usepackage{booktabs} % For formal tables

\usepackage{xcolor}
\usepackage{amsmath}
\usepackage{algorithm}
\usepackage[noend]{algpseudocode}
\makeatletter
\def\BState{\State\hskip-\ALG@thistlm}
\makeatother

\newcommand\TODO[1]{\textcolor{red}{#1}}
\newcommand{\lucenequery}[1]{\vspace{1mm}\texttt{#1}\vspace{1mm}}
% Copyright
%\setcopyright{none}
%\setcopyright{acmcopyright}
%\setcopyright{acmlicensed}
\setcopyright{rightsretained}
%\setcopyright{usgov}
%\setcopyright{usgovmixed}
%\setcopyright{cagov}
%\setcopyright{cagovmixed}


% DOI
\acmDOI{10.475/123_4}

% ISBN
\acmISBN{123-4567-24-567/08/06}

%Conference
\acmConference[WOODSTOCK'97]{ACM Woodstock conference}{July 1997}{El
  Paso, Texas USA} 
\acmYear{1997}
\copyrightyear{2016}

\acmPrice{15.00}


\begin{document}
\title{Latency Reduction via Decision Tree Based Query Construction}
\author{Aman Grover, Dhruv Arya, Ganesh Venkataraman\\ LinkedIn Corporation}
\email{jsmith@affiliation.org}

\begin{abstract}
LinkedIn as a professional network serves the career needs of 450 Million plus members. 
The task of job recommendation system is to find the suitable job among a corpus of several million jobs and serve this in real time under tight latency constraints. 
Job search involves finding suitable job listings given a user, query and
context.  Typical scoring function for both search and recommendations 
involves evaluating a function that matches various fields in the job description 
with various fields in the member profile.
This in turn translates to evaluating a function with several thousands of features to get the right ranking. 
In recommendations, evaluating all the jobs in the corpus for all members is not possible given the latency constraints. 
On the other hand, reducing the candidate set could 
potentially involve loss of relevant jobs. 
This work provides a novel way of improving the candidate jobs for the job seeker without hurting the overall relevance of the product. 
We propose a novel way to model the underlying complex ranking function with a
decision tree. The variable within the branching of the decision tree would be
candidate query clauses. We developed an offline framework which
evaluates the quality of the decision tree with respect to latency and recall.
We tested the approach on job search and recommendations on LinkedIn and A/B tests show significant improvements in member engagement and latency. 
Our techniques helped reduce job search latency by over {\bf 67\%} and our
recommendations latency by over {\bf 55\%}.
{\bf As of writing the approach powers all of job search and recommendations on LinkedIn.} 
\end{abstract}

\maketitle



\author{Julius P.~Kumquat }
\affiliation{\institution{The Kumquat Consortium}}
\email{jpkumquat@consortium.net}

% The default list of authors is too long for headers}
\renewcommand{\shortauthors}{B. Trovato et al.}



%
% The code below should be generated by the tool at
% http://dl.acm.org/ccs.cfm
% Please copy and paste the code instead of the example below. 
%
\begin{CCSXML}
 <ccs2012>
 <concept>
  <concept_id>10010520.10010553.10010562</concept_id>
  <concept_desc>Information retrieval~Document filtering</concept_desc>
  <concept_desc>Recommender Systems~Query Reformulation</concept_desc>
  <concept_desc>Machine Learning~Decision Trees</concept_desc>
  <concept_significance>500</concept_significance>
 </concept>
 <concept>
  <concept_id>10010520.10010575.10010755</concept_id>
  <concept_desc>Computer systems organization~Redundancy</concept_desc>
  <concept_significance>300</concept_significance>
 </concept>
 <concept>
  <concept_id>10010520.10010553.10010554</concept_id>
  <concept_desc>Computer systems organization~Robotics</concept_desc>
  <concept_significance>100</concept_significance>
 </concept>
 <concept>
  <concept_id>10003033.10003083.10003095</concept_id>
  <concept_desc>Networks~Network reliability</concept_desc>
  <concept_significance>100</concept_significance>
 </concept>
</ccs2012>  
\end{CCSXML}

\ccsdesc[500]{Computer systems organization~Embedded systems}
\ccsdesc[300]{Computer systems organization~Redundancy}
\ccsdesc{Computer systems organization~Robotics}
\ccsdesc[100]{Networks~Network reliability}

% We no longer use \terms command
%\terms{Theory}

\keywords{ACM procng}





\bibliographystyle{abbr}
\bibliography{sigproc} 

\end{document}
